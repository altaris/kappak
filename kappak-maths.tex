%% The MIT License (MIT)
%% 
%% Copyright (c) 2014 Cédric Ho Thanh
%% 
%% Permission is hereby granted, free of charge, to any person obtaining a copy
%% of this software and associated documentation files (the "Software"), to deal
%% in the Software without restriction, including without limitation the rights
%% to use, copy, modify, merge, publish, distribute, sublicense, and/or sell
%% copies of the Software, and to permit persons to whom the Software is
%% furnished to do so, subject to the following conditions:
%% 
%% The above copyright notice and this permission notice shall be included in all
%% copies or substantial portions of the Software.
%% 
%% THE SOFTWARE IS PROVIDED "AS IS", WITHOUT WARRANTY OF ANY KIND, EXPRESS OR
%% IMPLIED, INCLUDING BUT NOT LIMITED TO THE WARRANTIES OF MERCHANTABILITY,
%% FITNESS FOR A PARTICULAR PURPOSE AND NONINFRINGEMENT. IN NO EVENT SHALL THE
%% AUTHORS OR COPYRIGHT HOLDERS BE LIABLE FOR ANY CLAIM, DAMAGES OR OTHER
%% LIABILITY, WHETHER IN AN ACTION OF CONTRACT, TORT OR OTHERWISE, ARISING FROM,
%% OUT OF OR IN CONNECTION WITH THE SOFTWARE OR THE USE OR OTHER DEALINGS IN THE
%% SOFTWARE.

%% Definition of mathematical commands

\ProvidesFile{kappak-maths.tex}[Part of kappak package]

%% ----------------------------------------
%% Operators
%% ----------------------------------------

\ifthenelse{\boolean{@b_maths_use_operators}}{

    \DeclareMathOperator {\Ann}   {Ann}
    \DeclareMathOperator {\Aut}   {Aut}
    \DeclareMathOperator {\Bil}   {Bil}
    \DeclareMathOperator {\car}   {char}
    \DeclareMathOperator {\card}  {card}
    \DeclareMathOperator {\Card}  {Card}
    \DeclareMathOperator {\codom} {codom}
    \DeclareMathOperator {\coker} {coker}
    \DeclareMathOperator*{\colim} {colim}
    \DeclareMathOperator {\dom}   {dom}
    \DeclareMathOperator {\Dom}   {Dom}
    \DeclareMathOperator {\End}   {end}
    \DeclareMathOperator {\END}   {End}
    \DeclareMathOperator {\Ext}   {Ext}
    \DeclareMathOperator {\ev}    {ev}
    \DeclareMathOperator {\Frac}  {Frac}
    \DeclareMathOperator {\GL}    {GL}
    \DeclareMathOperator {\Hom}   {Hom}
    \DeclareMathOperator {\id}    {id}
    \DeclareMathOperator {\Id}    {Id}
    \DeclareMathOperator {\im}    {im}
    \DeclareMathOperator {\Ind}   {Ind}
    \DeclareMathOperator {\Ker}   {Ker}
    \DeclareMathOperator {\Map}   {Map}
    \DeclareMathOperator {\Mor}   {Mor}
    \DeclareMathOperator {\ob}    {ob}
    \DeclareMathOperator {\Ob}    {Ob}
    \DeclareMathOperator {\Orb}   {Orb}
    \DeclareMathOperator {\proj}  {proj}
    \DeclareMathOperator {\rk}    {rk}
    \DeclareMathOperator {\SL}    {SL}
    \DeclareMathOperator {\Stab}  {Stab}
    \DeclareMathOperator {\Spec}  {Spec}
    \DeclareMathOperator {\Sym}   {Sym}
    \DeclareMathOperator {\Tor}   {Tor}
    \DeclareMathOperator {\tr}    {tr}
    
}{} % \ifthenelse{\boolean{@b_maths_use_operators}}{

%% ----------------------------------------
%% Categories
%% ----------------------------------------

\ifthenelse{\boolean{@b_maths_use_categories}}{

    \ifthenelse{\equal{\@categoryStyle}{bf}}{
        \newcommand{\@cat}{\mathbf}
    }{}

    \ifthenelse{\equal{\@categoryStyle}{script}}{
        \newcommand{\@cat}{\@cal}
    }{}

    \ifthenelse{\equal{\@categoryStyle}{rm}}{
        \newcommand{\@cat}{\mathrm}
    }{}

    \ifthenelse{\equal{\@categoryStyle}{sf}}{
        \newcommand{\@cat}{\mathsf}
    }{}

    \ifthenelse{\equal{\@categoryStyle}{tt}}{
        \newcommand{\@cat}{\mathtt}
    }{}

    \ifthenelse{\equal{\@categoryStyle}{script}}{
        \newcommand{\op}{\mathrm{op}}
        \newcommand{\newcategory}[4]{
            \newcommand{#1}{#2\@cat{#3}\!#4}
        }
    }{
        \newcommand{\op}{\@cat{op}}
        \newcommand{\newcategory}[4]{
            \newcommand{#1}{\@cat{#2#3#4}}
        }
    } % \ifthenelse{\equal{\@categoryStyle}{script}}{

%% --------------------
%% Some categories
%% --------------------
    
    \newcategory{\Ab}    {}{A}{b}
    \newcategory{\Alg}   {}{A}{lg}
    \newcategory{\Cat}   {}{C}{at}
    \newcategory{\CAT}   {}{C}{AT}
    \newcategory{\Ch}    {}{C}{h}
    \newcategory{\Fin}   {}{F}{in}
    \newcategory{\Graph} {}{G}{raph}
    \newcategory{\Grp}   {}{G}{rp}
    \newcategory{\Mod}   {}{M}{od}
    \newcategory{\Mon}   {}{M}{on}
    \newcategory{\Op}    {}{O}{p}
    \newcategory{\Pos}   {}{P}{os}
    \newcategory{\pSh}   {p}{S}{h}
    \newcategory{\Ring}  {}{R}{ing}
    \newcategory{\Set}   {}{S}{et}
    \newcategory{\SET}   {}{S}{ET}
    \newcategory{\Sh}    {}{S}{h}
    \newcategory{\sSet}  {s}{S}{et}
    \newcategory{\Top}   {}{T}{op}
    \newcategory{\Vect}  {}{V}{ect}
    
%% --------------------
%% cat style
%% --------------------

    \newcommand{\catA}{\@cat{a}}
    \newcommand{\catB}{\@cat{b}}
    \newcommand{\catC}{\@cat{c}}
    \newcommand{\catD}{\@cat{d}}
    \newcommand{\catE}{\@cat{e}}
    \newcommand{\catF}{\@cat{f}}
    \newcommand{\catG}{\@cat{g}}
    \newcommand{\catH}{\@cat{h}}
    \newcommand{\catI}{\@cat{i}}
    \newcommand{\catJ}{\@cat{j}}
    \newcommand{\catK}{\@cat{k}}
    \newcommand{\catL}{\@cat{l}}
    \newcommand{\catM}{\@cat{m}}
    \newcommand{\catN}{\@cat{n}}
    \newcommand{\catO}{\@cat{o}}
    \newcommand{\catP}{\@cat{p}}
    \newcommand{\catQ}{\@cat{q}}
    \newcommand{\catR}{\@cat{r}}
    \newcommand{\catS}{\@cat{s}}
    \newcommand{\catT}{\@cat{t}}
    \newcommand{\catU}{\@cat{u}}
    \newcommand{\catV}{\@cat{v}}
    \newcommand{\catW}{\@cat{w}}
    \newcommand{\catX}{\@cat{x}}
    \newcommand{\catY}{\@cat{y}}
    \newcommand{\catZ}{\@cat{z}}
    \newcommand{\catAA}{\@cat{A}}
    \newcommand{\catBB}{\@cat{B}}
    \newcommand{\catCC}{\@cat{C}}
    \newcommand{\catDD}{\@cat{D}}
    \newcommand{\catEE}{\@cat{E}}
    \newcommand{\catFF}{\@cat{F}}
    \newcommand{\catGG}{\@cat{G}}
    \newcommand{\catHH}{\@cat{H}}
    \newcommand{\catII}{\@cat{I}}
    \newcommand{\catJJ}{\@cat{J}}
    \newcommand{\catKK}{\@cat{K}}
    \newcommand{\catLL}{\@cat{L}}
    \newcommand{\catMM}{\@cat{M}}
    \newcommand{\catNN}{\@cat{N}}
    \newcommand{\catOO}{\@cat{O}}
    \newcommand{\catPP}{\@cat{P}}
    \newcommand{\catQQ}{\@cat{Q}}
    \newcommand{\catRR}{\@cat{R}}
    \newcommand{\catSS}{\@cat{S}}
    \newcommand{\catTT}{\@cat{T}}
    \newcommand{\catUU}{\@cat{U}}
    \newcommand{\catVV}{\@cat{V}}
    \newcommand{\catWW}{\@cat{W}}
    \newcommand{\catXX}{\@cat{X}}
    \newcommand{\catYY}{\@cat{Y}}
    \newcommand{\catZZ}{\@cat{Z}}
    
}{} % \ifthenelse{\boolean{@b_maths_use_categories}}{

%% ----------------------------------------
%% Arrows
%% ----------------------------------------

\ifthenelse{\boolean{@b_maths_use_arrows}}{

    \newcommand{\incl}{\hookrightarrow}
    \newcommand{\leftincl}{\hookleftarrow}
    \newcommand{\longincl}{\lhook\joinrel\relbar\joinrel\rightarrow}
    \newcommand{\longleftincl}{\leftarrow\joinrel\relbar\joinrel\rhook}
	\newcommand{\xincl}{\xhookrightarrow}
	\newcommand{\xleftincl}{\xhookleftarrow}    
    
	\renewcommand{\twoheadrightarrow}{\rightarrow\mathrel{\mkern-14mu}\rightarrow}
	\renewcommand{\twoheadleftarrow}{\leftarrow\mathrel{\mkern-14mu}\leftarrow}	
	\newcommand{\longtwoheadrightarrow}{\relbar\joinrel\twoheadrightarrow}
	\newcommand{\longtwoheadleftarrow}{\twoheadleftarrow\joinrel\relbar}	
	\newcommand{\xtwoheadrightarrow}[2][]{
	  \xrightarrow[#1]{#2}\mathrel{\mkern-14mu}\rightarrow
	}
	\newcommand{\xtwoheadleftarrow}[2][]{
	  \leftarrow\mathrel{\mkern-14mu}\xleftarrow[#1]{#2}
	}    
    
    \newcommand{\epi}{\twoheadrightarrow}
    \newcommand{\leftepi}{\twoheadleftarrow}
    \newcommand{\longepi}{\longtwoheadrightarrow}
    \newcommand{\longleftepi}{\longtwoheadleftarrow}
    \newcommand{\xepi}{\xtwoheadrightarrow}
    \newcommand{\xleftepi}{\xtwoheadleftarrow}
    
    \newcommand{\mono}{\rightarrowtail}
    \newcommand{\longmono}{
        \stackrel{
            \begin{tikzpicture}
                \draw[>->] (0, 0) -- (0.55, 0);
            \end{tikzpicture}
        }{}
    }
    
    \newcommand{\longmultimap}{\relbar\joinrel\multimap}
    
}{} % \ifthenelse{\boolean{@b_maths_use_arrows}}{

%% ----------------------------------------
%% Other / handy
%% ----------------------------------------

\ifthenelse{\boolean{@b_maths_use_misc}}{

    \newcommand{\bs}{\setminus}
    \newcommand{\es}{\emptyset}
    \newcommand{\ox}{\otimes}
    \newcommand{\x}{\times}
    
    \renewcommand{\epsilon}{\varepsilon}
    
    \ifthenelse{\boolean{@b_extpack_use_xecjk}}{
    	\newcommand{\yoneda}{\text{よ}}
    }{
	    \newcommand{\yoneda}{\mathsf{y}}
    }
    
    % Define custom big operators, whose size are that of \sum
    % 9000+ thanks to egreg, https://tex.stackexchange.com/questions/23432/how-to-create-my-own-math-operator-with-limits
    \DeclareRobustCommand\@bigop[1]{
    	\mathop{\vphantom{\sum}\mathpalette\@bigop@{#1}}\slimits@
    }
    \newcommand{\@bigop@}[2]{
    	\vcenter{
    		\sbox\z@{$#1\sum$}
    		\hbox{\resizebox{\ifx#1\displaystyle.9\fi\dimexpr\ht\z@+\dp\z@}{!}{$\m@th#2$}}
    	}
    }
    \newcommand{\DeclareBigOperator}[2]{\newcommand{#1}{\DOTSB\@bigop{#2}}}
    
    % Defines a rotated symbol that sizes properly
    \newcommand{\DeclareRotatedSymbol}[3]{
    	\newcommand{#1}{{
    		\mathchoice{
    			\rotatebox[origin=c]{#2}{$\displaystyle{#3}$}
    		}{
    			\rotatebox[origin=c]{#2}{$\textstyle{#3}$}
    		}{
	    		\rotatebox[origin=c]{#2}{$\scriptstyle{#3}$}
	    	}{
		    	\rotatebox[origin=c]{#2}{$\scriptscriptstyle{#3}$}
		    }}
		}
	}
    
}{} % \ifthenelse{\boolean{@b_maths_use_misc}}{
