%% The MIT License (MIT)
%% 
%% Copyright (c) 2014 Cédric Ho Thanh
%% 
%% Permission is hereby granted, free of charge, to any person obtaining a copy
%% of this software and associated documentation files (the "Software"), to deal
%% in the Software without restriction, including without limitation the rights
%% to use, copy, modify, merge, publish, distribute, sublicense, and/or sell
%% copies of the Software, and to permit persons to whom the Software is
%% furnished to do so, subject to the following conditions:
%% 
%% The above copyright notice and this permission notice shall be included in all
%% copies or substantial portions of the Software.
%% 
%% THE SOFTWARE IS PROVIDED "AS IS", WITHOUT WARRANTY OF ANY KIND, EXPRESS OR
%% IMPLIED, INCLUDING BUT NOT LIMITED TO THE WARRANTIES OF MERCHANTABILITY,
%% FITNESS FOR A PARTICULAR PURPOSE AND NONINFRINGEMENT. IN NO EVENT SHALL THE
%% AUTHORS OR COPYRIGHT HOLDERS BE LIABLE FOR ANY CLAIM, DAMAGES OR OTHER
%% LIABILITY, WHETHER IN AN ACTION OF CONTRACT, TORT OR OTHERWISE, ARISING FROM,
%% OUT OF OR IN CONNECTION WITH THE SOFTWARE OR THE USE OR OTHER DEALINGS IN THE
%% SOFTWARE.

%% Definitions of environments

\ProvidesFile{kappak-environments.tex}[Part of kappak package]

%% ----------------------------------------
%% Environments
%% ----------------------------------------

\ifthenelse{\boolean{@b_maths_use_theoremEnvs}}{

    \ifthenelse{\equal{\@langage}{french}}{
    
        \def\@assertionName         {Assertion}
        \def\@assertionsName        {Assertions}
        \def\@axiomName             {Axiome}
        \def\@axiomsName            {Axiomes}
        \def\@conjectureName        {Conjecture}
        \def\@conjecturesName       {Conjectures}
        \def\@conventionName        {Convention}
        \def\@conventionsName       {Conventions}
        \def\@corollaryName         {Corollaire}
        \def\@corollariesName       {Corollaires}
        \def\@definitionName        {Définition}
        \def\@definitionsName       {Définitions}
        \def\@exampleName           {Exemple}
        \def\@examplesName          {Exemples}
        \def\@exerciseName          {Exercice}
        \def\@exercisesName         {Exercices}
        \def\@lemmaName             {Lemme}
        \def\@lemmasName            {Lemmes}
        \def\@notationName          {Notation}
        \def\@notationsName         {Notations}
        \def\@proofName             {Démonstration}
        \def\@proofsName            {Démonstrations}
        \def\@propertyName          {Propriété}
        \def\@propertiesName        {Propriétés}
        \def\@propositionName       {Proposition}
        \def\@propositionsName      {Propositions}
        \def\@questionName          {Question}
        \def\@questionsName         {Questions}
        \def\@remarkName            {Remarque}
        \def\@remarksName           {Remarques}
        \def\@reminderName          {Rappel}
        \def\@remindersName         {Rappels}
        \def\@scholiaName           {Scholie}
        \def\@scholiasName          {Scholies}
        \def\@solutionName          {Solution}
        \def\@solutionsName         {Solutions}
        \def\@terminologyName       {Terminologie}
        \def\@terminologiesName     {Terminologies}
        \def\@theoremName           {Théorème}
        \def\@theoremsName          {Théorèmes}
        
    }{} % \ifthenelse{\equal{\@langage}{french}}{
    
    \ifthenelse{\equal{\@langage}{english}}{
    
        \def\@assertionName         {Assertion}
        \def\@assertionsName        {Assertions}
        \def\@axiomName             {Axiom}
        \def\@axiomsName            {Axioms}
        \def\@conjectureName        {Conjecture}
        \def\@conjecturesName       {Conjectures}
        \def\@conventionName        {Convention}
        \def\@conventionsName       {Conventions}
        \def\@corollaryName         {Corollary}
        \def\@corollariesName       {Corollaries}
        \def\@definitionName        {Definition}
        \def\@definitionsName       {Definitions}
        \def\@exampleName           {Example}
        \def\@examplesName          {Examples}
        \def\@exerciseName          {Exercise}
        \def\@exercisesName         {Exercises}
        \def\@lemmaName             {Lemma}
        \def\@lemmasName            {Lemmas}
        \def\@notationName          {Notation}
        \def\@notationsName         {Notations}
        \def\@proofName             {Proof}
        \def\@proofsName            {Proofs}
        \def\@propertyName          {Property}
        \def\@propertiesName        {Properties}
        \def\@propositionName       {Proposition}
        \def\@propositionsName      {Propositions}
        \def\@questionName          {Question}
        \def\@questionsName         {Questions}
        \def\@remarkName            {Remark}
        \def\@remarksName           {Remarks}
        \def\@reminderName          {Reminder}
        \def\@remindersName         {Reminders}
        \def\@scholiaName           {Scholia}
        \def\@scholiasName          {Scholias}
        \def\@solutionName          {Solution}
        \def\@solutionsName         {Solutions}
        \def\@theoremName           {Theorem}
        \def\@theoremsName          {Theorems}
        \def\@terminologyName       {Terminology}
        \def\@terminologiesName     {Terminologies}

    }{} % \ifthenelse{\equal{\@langage}{english}}{
    
    \ifthenelse{\boolean{@b_maths_swapnumbers}}{
        \swapnumbers
    }{}
    
    
    \ifthenelse{\boolean{@b_extpack_use_beamer}}{
    
        \theoremstyle{definition}
        
        \newtheorem{baxiom}                     {\@axiomName}       [section]
        \newtheorem{baxioms}        [baxiom]    {\@axiomsName}
        \newtheorem*{baxiom*}                   {\@axiomName}
        \newtheorem*{baxioms*}                  {\@axiomsName}
        \newtheorem{bdefinition}    [baxiom]    {\@definitionName}
        \newtheorem{bdefinitions}   [baxiom]    {\@definitionsName}
        \newtheorem*{bdefinition*}              {\@definitionName}
        \newtheorem*{bdefinitions*}             {\@definitionsName}
        \newtheorem{bexample}       [baxiom]    {\@exampleName}
        \newtheorem{bexamples}      [baxiom]    {\@examplesName}
        \newtheorem*{bexample*}                 {\@exampleName}
        \newtheorem*{bexamples*}                {\@examplesName}
        \newtheorem{bexercise}      [baxiom]    {\@exerciseName}
        \newtheorem{bexercises}     [baxiom]    {\@exercisesName}
        \newtheorem*{bexercise*}                {\@exerciseName}
        \newtheorem*{bexercises*}               {\@exercisesName}
        
        \ifthenelse{\boolean{@b_maths_use_theoremStyles}}{
            \theoremstyle{plain}
        }{}
        
        \newtheorem{bassertion}     [baxiom]    {\@assertionName}
        \newtheorem{bassertions}    [baxiom]    {\@assertionsName}
        \newtheorem*{bassertion*}               {\@assertionName}
        \newtheorem*{bassertions*}              {\@assertionsName}
        \newtheorem{bconjecture}    [baxiom]    {\@conjectureName}
        \newtheorem{bconjectures}   [baxiom]    {\@conjecturesName}
        \newtheorem*{bconjecture*}              {\@conjectureName}
        \newtheorem*{bconjectures*}             {\@conjecturesName}
        \newtheorem{bcorollary}     [baxiom]    {\@corollaryName}
        \newtheorem{bcorollaries}   [baxiom]    {\@corollariesName}
        \newtheorem*{bcorollary*}               {\@corollaryName}
        \newtheorem*{bcorollaries*}             {\@corollariesName}
        \newtheorem{blemma}         [baxiom]    {\@lemmaName}
        \newtheorem{blemmas}        [baxiom]    {\@lemmasName}
        \newtheorem*{blemma*}                   {\@lemmaName}
        \newtheorem*{blemmas*}                  {\@lemmasName}
        \newtheorem{bproperty}      [baxiom]    {\@propertyName}
        \newtheorem{bproperties}    [baxiom]    {\@propertiesName}
        \newtheorem*{bproperty*}                {\@propertyName}
        \newtheorem*{bproperties*}              {\@propertiesName}
        \newtheorem{bproposition}   [baxiom]    {\@propositionName}
        \newtheorem{bpropositions}  [baxiom]    {\@propositionsName}
        \newtheorem*{bproposition*}             {\@propositionName}
        \newtheorem*{bpropositions*}            {\@propositionsName}
        \newtheorem{bscholia}       [baxiom]    {\@scholiaName}
        \newtheorem{bscholias}      [baxiom]    {\@scholiasName}
        \newtheorem{bscholia*}                  {\@scholiaName}
        \newtheorem{bscholias*}                 {\@scholiasName}
        \newtheorem{btheorem}       [baxiom]    {\@theoremName}
        \newtheorem{btheorems}      [baxiom]    {\@theoremsName}
        \newtheorem*{btheorem*}                 {\@theoremName}
        \newtheorem*{btheorems*}                {\@theoremsName}
        
        \ifthenelse{\boolean{@b_maths_use_theoremStyles}}{
            \theoremstyle{remark}
        }{}
        
        \newtheorem{bconvention}    [baxiom]    {\@conventionName}
        \newtheorem{bconventions}   [baxiom]    {\@conventionsName}
        \newtheorem*{bconvention*}              {\@conventionName}
        \newtheorem*{bconventions*}             {\@conventionsName}
        \newtheorem{bnotation}      [baxiom]    {\@notationName}
        \newtheorem{bnotations}     [baxiom]    {\@notationsName}
        \newtheorem*{bnotation*}                {\@notationName}
        \newtheorem*{bnotations*}               {\@notationsName}       
        \newtheorem{bremark}        [baxiom]    {\@remarkName}
        \newtheorem{bremarks}       [baxiom]    {\@remarksName}
        \newtheorem*{bremark*}                  {\@remarkName}
        \newtheorem*{bremarks*}                 {\@remarksName}
        \newtheorem{breminder}      [baxiom]    {\@reminderName}
        \newtheorem{breminders}     [baxiom]    {\@remindersName}
        \newtheorem*{breminder*}                {\@reminderName}
        \newtheorem*{breminders*}               {\@remindersName}
        \newtheorem{bterminology}   [baxiom]    {\@terminologyName}
        \newtheorem{bterminologies} [baxiom]    {\@terminologiesName}
        \newtheorem*{bterminology*}             {\@terminologyName}
        \newtheorem*{bterminologies*}           {\@terminologiesName}
        
    }{ % \ifthenelse{\boolean{@b_extpack_use_beamer}}{
    
        \theoremstyle{definition}
        
        \newtheorem{axiom}                      {\@axiomName}       [section]
        \newtheorem{axioms}         [axiom]     {\@axiomsName}
        \newtheorem*{axiom*}                    {\@axiomName}
        \newtheorem*{axioms*}                   {\@axiomsName}
        \newtheorem{definition}     [axiom]     {\@definitionName}
        \newtheorem{definitions}    [axiom]     {\@definitionsName}
        \newtheorem*{definition*}               {\@definitionName}
        \newtheorem*{definitions*}              {\@definitionsName}
        \newtheorem{example}        [axiom]     {\@exampleName}
        \newtheorem{examples}       [axiom]     {\@examplesName}
        \newtheorem*{example*}                  {\@exampleName}
        \newtheorem*{examples*}                 {\@examplesName}
        \newtheorem{exercise}       [axiom]     {\@exerciseName}
        \newtheorem{exercises}      [axiom]     {\@exercisesName}
        \newtheorem*{exercise*}                 {\@exerciseName}
        \newtheorem*{exercises*}                {\@exercisesName}
        \newtheorem{question}       [axiom]     {\@questionName}
        \newtheorem{questions}      [axiom]     {\@questionsName}
        \newtheorem*{question*}                 {\@questionName}
        \newtheorem*{questions*}                {\@questionsName}
        
        \ifthenelse{\boolean{@b_maths_use_theoremStyles}}{
            \theoremstyle{plain}
        }{}
        
        \newtheorem{assertion}      [axiom]     {\@assertionName}
        \newtheorem{assertions}     [axiom]     {\@assertionsName}
        \newtheorem*{assertion*}                {\@assertionName}
        \newtheorem*{assertions*}               {\@assertionsName}
        \newtheorem{conjecture}     [axiom]     {\@conjectureName}
        \newtheorem{conjectures}    [axiom]     {\@conjecturesName}
        \newtheorem*{conjecture*}               {\@conjectureName}
        \newtheorem*{conjectures*}              {\@conjecturesName}
        \newtheorem{corollary}      [axiom]     {\@corollaryName}
        \newtheorem{corollaries}    [axiom]     {\@corollariesName}
        \newtheorem*{corollary*}                {\@corollaryName}
        \newtheorem*{corollaries*}              {\@corollariesName}
        \newtheorem{lemma}          [axiom]     {\@lemmaName}
        \newtheorem{lemmas}         [axiom]     {\@lemmasName}
        \newtheorem*{lemma*}                    {\@lemmaName}
        \newtheorem*{lemmas*}                   {\@lemmasName}
        \newtheorem{property}       [axiom]     {\@propertyName}
        \newtheorem{properties}     [axiom]     {\@propertiesName}
        \newtheorem*{property*}                 {\@propertyName}
        \newtheorem*{properties*}               {\@propertiesName}
        \newtheorem{proposition}    [axiom]     {\@propositionName}
        \newtheorem{propositions}   [axiom]     {\@propositionsName}
        \newtheorem*{proposition*}              {\@propositionName}
        \newtheorem*{propositions*}             {\@propositionsName}
        \newtheorem{scholia}        [axiom]     {\@scholiaName}
        \newtheorem{scholias}       [axiom]     {\@scholiasName}
        \newtheorem{scholia*}                   {\@scholiaName}
        \newtheorem{scholias*}                  {\@scholiasName}
        \newtheorem{theorem}        [axiom]     {\@theoremName}
        \newtheorem{theorems}       [axiom]     {\@theoremsName}
        \newtheorem*{theorem*}                  {\@theoremName}
        \newtheorem*{theorems*}                 {\@theoremsName}
        
        \ifthenelse{\boolean{@b_maths_use_theoremStyles}}{
            \theoremstyle{remark}
        }{}
        
        \newtheorem{convention}     [axiom]     {\@conventionName}
        \newtheorem{conventions}    [axiom]     {\@conventionsName}
        \newtheorem*{convention*}               {\@conventionName}
        \newtheorem*{conventions*}              {\@conventionsName}
        \newtheorem{notation}       [axiom]     {\@notationName}
        \newtheorem{notations}      [axiom]     {\@notationsName}
        \newtheorem*{notation*}                 {\@notationName}
        \newtheorem*{notations*}                {\@notationsName}
        \newtheorem{remark}         [axiom]     {\@remarkName}
        \newtheorem{remarks}        [axiom]     {\@remarksName}
        \newtheorem*{remark*}                   {\@remarkName}
        \newtheorem*{remarks*}                  {\@remarksName}
        \newtheorem{reminder}       [axiom]     {\@reminderName}
        \newtheorem{reminders}      [axiom]     {\@remindersName}
        \newtheorem*{reminder*}                 {\@reminderName}
        \newtheorem*{reminders*}                {\@remindersName}
        \newtheorem{terminology}    [axiom]     {\@terminologyName}
        \newtheorem{terminologies}  [axiom]     {\@terminologiesName}
        \newtheorem*{terminology*}              {\@terminologyName}
        \newtheorem*{terminologies*}            {\@terminologiesName}
        
    } % \ifthenelse{\boolean{@b_extpack_use_beamer}}{
    
        %% solution(s) and proof(s)
    \ifthenelse{\boolean{@b_extpack_use_beamer}}{
    
        \renewenvironment{bproof}[1][\@proofName]{
            \par
            \pushQED{\qed}
            \normalfont \topsep6\p@\@plus6\p@\relax
            \trivlist
            \item[\hskip\labelsep
                \itshape
            #1\@addpunct{.}]\ignorespaces
        }{
            \popQED\endtrivlist\@endpefalse
        }
        \newenvironment{bproofs}[1][\@proofsName]{
            \par
            \pushQED{\qed}
            \normalfont \topsep6\p@\@plus6\p@\relax
            \trivlist
            \item[\hskip\labelsep
                \itshape
            #1\@addpunct{.}]\ignorespaces
        }{
            \popQED\endtrivlist\@endpefalse
        }
        \newenvironment{bsolution}[1][\@solutionName]{
            \par
            \pushQED{\qed}
            \normalfont \topsep6\p@\@plus6\p@\relax
            \trivlist
            \item[\hskip\labelsep
                \itshape
            #1\@addpunct{.}]\ignorespaces
        }{
            \hfill\ensuremath{\blacksquare}\endtrivlist\@endpefalse
        }
        \newenvironment{bsolutions}[1][\@solutionsName]{
            \par
            \pushQED{\qed}
            \normalfont \topsep6\p@\@plus6\p@\relax
            \trivlist
            \item[\hskip\labelsep
                \itshape
            #1\@addpunct{.}]\ignorespaces
        }{
            \hfill\ensuremath{\blacksquare}\endtrivlist\@endpefalse
        }
        
    }{
    
        \renewenvironment{proof}[1][\@proofName]{
            \par
            \pushQED{\qed}
            \normalfont \topsep6\p@\@plus6\p@\relax
            \trivlist
            \item[\hskip\labelsep
                \itshape
            #1\@addpunct{.}]\ignorespaces
        }{
            \popQED\endtrivlist\@endpefalse
        }
        \newenvironment{proofs}[1][\@proofsName]{
            \par
            \pushQED{\qed}
            \normalfont \topsep6\p@\@plus6\p@\relax
            \trivlist
            \item[\hskip\labelsep
                \itshape
            #1\@addpunct{.}]\ignorespaces
        }{
            \popQED\endtrivlist\@endpefalse
        }
        \newenvironment{solution}[1][\@solutionName]{
            \par
            \pushQED{\qed}
            \normalfont \topsep6\p@\@plus6\p@\relax
            \trivlist
            \item[\hskip\labelsep
                \itshape
            #1\@addpunct{.}]\ignorespaces
        }{
            \hfill\ensuremath{\blacksquare}\endtrivlist\@endpefalse
        }
        \newenvironment{solutions}[1][\@solutionsName]{
            \par
            \pushQED{\qed}
            \normalfont \topsep6\p@\@plus6\p@\relax
            \trivlist
            \item[\hskip\labelsep
                \itshape
            #1\@addpunct{.}]\ignorespaces
        }{
            \hfill\ensuremath{\blacksquare}\endtrivlist\@endpefalse
        }
    } % \ifthenelse{\boolean{@b_extpack_use_beamer}}{
    
}{} % \ifthenelse{\boolean{@b_maths_use_theoremEnvs}}{

\newenvironment{dedication}{
    \cleardoublepage
    \thispagestyle{empty}
    \vspace*{\stretch{1}}
    \hfill\begin{minipage}[t]{0.66\textwidth}
    \raggedright
}{
    \end{minipage}
    \vspace*{\stretch{3}}
    \clearpage
} % \newenvironment{dedication}{